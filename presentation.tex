%-------------------------------------------------------------------------------
% Main file of presentation
%-------------------------------------------------------------------------------

\documentclass[t]{beamer}

%-------------------------------------------------------------------------------
% pdfLaTeX (not used)
%\usepackage[T1]{fontenc}
%\usepackage[utf8]{inputenc}
%\usepackage[american]{babel} %[ngerman]


%-------------------------------------------------------------------------------
% XeLaTeX (used)
\usepackage{polyglossia} % replaces babel
%\setmainlanguage[spelling=new,babelshorthands=true]{german}
%\usepackage{fontspec}
%\usepackage{relsize}
%\usepackage{microtype} % better management of overfulls

%-------------------------------------------------------------------------------
% beamer presentation theme (local to this project)
%\usetheme{beamerthemefau-4-3}
\usepackage{beamertheme/beamerthemefau-4-3}

%-------------------------------------------------------------------------------
% Quotes
%\usepackage[babel=once,german=quotes]{csquotes}

%-------------------------------------------------------------------------------
% strike out text with \st{some text}
\usepackage{soul}

%-------------------------------------------------------------------------------
% Images, captions
\usepackage{graphicx}
%\usepackage{subfig}
\usepackage{wrapfig}
\usepackage{caption}
\usepackage{subcaption}
%\DeclareCaptionFont{mysize}{\fontsize{14}{9.6}\selectfont}
%\captionsetup[sub]{font=mysize,labelfont={bf,sf}}
%\captionsetup{font=mysize,labelfont={bf,sf}}

%-------------------------------------------------------------------------------
% non-floating float environment, figure H option
\usepackage{float}
\usepackage{scrhack}

%-------------------------------------------------------------------------------
% Include text in graphics
\usepackage{psfrag}

%-------------------------------------------------------------------------------
% math, symbols, math font, units
\usepackage{amsmath}
\usepackage{amsbsy}
\usepackage{amssymb}
\usepackage{xfrac}
\usepackage{bm}
\usepackage{cancel}
\usepackage{commath} % provides differential operators
\usepackage{textcomp} % required by gensymb
\usepackage{gensymb} % eg. \degree symbol
\usepackage{mathtools} % eg. \underbrace

\usepackage[math-style=ISO]{unicode-math} % font (XeLaTeX)
%\usepackage{upgreek} % (non-italic print) greek letters

\usepackage{custom_math} % local to this project
\usepackage{units} % local to this project

%-------------------------------------------------------------------------------
% line break in citations
\usepackage{breakcites}

%-------------------------------------------------------------------------------
% colored and shaded boxes
\usepackage{color, framed}

%-------------------------------------------------------------------------------
% theorem environment
\usepackage{theorem}

%-------------------------------------------------------------------------------
% tables with page break or flexible columns/rows
\usepackage{tabu}
\usepackage{longtable} % für Tabellen mit Seitenumbruch
\usepackage{multirow}
\usepackage{multicol}
\usepackage{makecell}
\usepackage{booktabs}
\usepackage{tabularx}

%-------------------------------------------------------------------------------
% make labels visible
%\usepackage{showkeys}

%-------------------------------------------------------------------------------
% links
\usepackage{hyperref}

%-------------------------------------------------------------------------------
% enumerate, itemize spacing
\usepackage{enumitem}
\setlist{  
  listparindent=\parindent,
  parsep=3pt, % space above each item
  topsep=2pt, % space above (all items together)
}
\setlist[itemize,1]{label=$\circ$}

% fix compatibility with beamer
\setitemize{label=\protect\usebeamerfont{itemize item}
                  \protect\usebeamercolor[fg]{itemize item}}
\setenumerate[1]{label=\protect\usebeamerfont{enumerate item}
                       \protect\usebeamercolor[fg]{enumerate item}
                       \insertenumlabel.}


%-------------------------------------------------------------------------------
% custom hyphenation
% e.g. write Neumann\hyp{}Randbedingung
\usepackage{hyphenat}


%-------------------------------------------------------------------------------
% setup title page
\title[]{Elasticity of one-dimensional continua and nanostructures}
\author{Prof. Ajeet Kumar, Prakhar Gupta, Smriti Singh}
\date{Erlangen, Summer term 2017}
\institute{Chair of Applied Mechanics}


%-------------------------------------------------------------------------------
% begin body of document and render title page

\begin{document}

% plain-option deactivates header and footer
\frame[plain,c]{\titlepage}



%-------------------------------------------------------------------------------
% include section introduction
%===============================================================================
\section*{Introduction}


%-------------------------------------------------------------------------------
\begin{frame}
  \frametitle{Extension, torsion and bending of rods/beams}
  
  \begin{tabularx}{\linewidth}{XX}
    {
    \begin{alignat*}{1}
      F &= E \, A \, \epsilon \\
        %&= E \, A \, \frac{\Delta l}{l}
    \end{alignat*}
    } & {
    \null
    } \\
    %\hline
    
    {
    \begin{alignat*}{1}
      M_t = G \, J \, \frac{\Theta}{L} \\
    \end{alignat*}
    } & {
    \null
    } \\
    %\hline
        
    {
    \begin{alignat*}{1}
      M = E \, I \, \kappa \\
    \end{alignat*}
    } & {
    \null
    }
  \end{tabularx}
  % TODO revisit this/these motivational slide/slides
  % put in three pictures
\end{frame}


%-------------------------------------------------------------------------------
\begin{frame}
  \frametitle{3D elasticity (recap of some basics)}
  
  \textbf{deformation map} $\underline{x} = \underline{f}(\underline{X})$
  \begin{itemize}
    \item takes a (Langrangian/material) vector/point $\underline{X}$ in the undeformed configuration
    \item returns a (Eulerian/spatial) vector/point $\underline{x}$ in the deformed configuration
  \end{itemize}
  \vspace{1em}
  
  \textbf{deformation gradient} $\doubleunderline{F}(\underline{X}) = \nabla_{\underline{X}} \,\underline{f}(\underline{X})$
  \begin{itemize}
    \item takes a (Langrangian/material) vector/point $\underline{X}$ in the undeformed configuration \newline
      $\rightarrow$ $\underline{X}$ is material point where the gradient is evaluated
    \item returns a second order (two-point) tensor $\doubleunderline{F}$ that depends on $\underline{X}$ (in the general case)
    \item $\doubleunderline{F}$ contains (first order accurate) information about how line elements in the material configuration are transformed into deformed (stretched and rotated) line elements in the spatial configuration by $\underline{f}(.)$ \newline
      $\rightarrow$ $\dif \underline{x} = \doubleunderline{F}(\underline{X}) \cdot \dif \underline{X}= \nabla_{\underline{X}} \,\underline{f}(\underline{X}) \cdot \dif \underline{X}$
  \end{itemize}
  \vspace{1em}
  
  \textbf{equations of equilibrium}
  \begin{itemize}
    \item material description: $\nabla_{\underline{X}} \cdot \doubleunderline{P} + \underline{B} = \rho_0 \cdot \underline{A}$
    \item spatial description: $\nabla_{\underline{x}} \cdot \doubleunderline{\sigma} + \underline{b} = \rho \cdot \underline{a}$
  \end{itemize}
\end{frame}


%-------------------------------------------------------------------------------
\begin{frame}
  \frametitle{1D elasticity}
  
  we talk about \textbf{slender bodies} ($\frac{\mathrm{Length}}{\mathrm{Diameter}} > 10$)
  \vspace{1em}
  
  we only have \textbf{one material coordinate} $s$ along the axis of the beam
  \begin{itemize}
    \item $s$ uniquely identifies a particular cross section along the beam axis
    \item model equations will be a system of ODEs in $s$ (instead of PDE)
    \item equations are obtained by averaging a PDE in 3 dimensions over the cross section
  \end{itemize}
  \vspace{1em}
  
  kinematic quantities in the model
  \begin{itemize}
    \item \textbf{position of centerline} $\underline{r}(s)$
    \item \textbf{orientation/rotation of cross section} $\doubleunderline{R}(s)$
    \item more accuracy $\rightarrow$ more unknowns
  \end{itemize}
\end{frame}




%-------------------------------------------------------------------------------
% section content of the presentation, overview, table of contents
\section*{Outline of this course}

% render table of contents page
\frame{
  % TODO TOC should not be spaced out over page
  \tableofcontents
}


%-------------------------------------------------------------------------------
% include section traditional_beams
%===============================================================================
\section{Traditional beam models}

%===============================================================================
\subsection{Euler-Bernoulli beam theory}

%-------------------------------------------------------------------------------
\begin{frame}
  \frametitle{The Euler-Bernoulli beam (planar case)}
  
  \begin{itemize}
    \item assumption of the model
      \begin{itemize}
        \item all cross sections remain planar
        \item cross section normal and centerline tangent are of same orientation $\varphi$ \newline
          \null \quad $\rightarrow$ model only allows for pure bending (no shear)
      \end{itemize}
    \item bending moment $M(s) = E \, I \, \kappa(s)$
    \item centerline of beam described by function $y(s)$
    \item curvature of centerline given by $\kappa(s) = \frac{\od[2]{y}{s}(s)}{\biggl( 1 + \bigl( \od{y}{s}(s) \bigr)^2 \biggr)^{\frac{3}{2}}}$ \newline
      \null \quad $\rightarrow$ geometric nonlinearity $\rightarrow$ non-linear ODE
    \item linear approximation for small deflections: $\od{y}{s} = \tan(\varphi) \ll 1 \Rightarrow \od{y}{s} \approx \varphi$
    \item Euler-Bernoulli beam equation: $M(s) = E \, I \, \od[2]{y}{s}$
  \end{itemize}
\end{frame}


%-------------------------------------------------------------------------------
\begin{frame}
  \frametitle{Example: Cantilever beam (part 1)}
  
  \begin{figure}
    \centering
    \includegraphics[width=16cm, keepaspectratio=true]{sections/traditional_beams/images/EulerCanitleverExample1part1}
  \end{figure}
  
  \begin{tabularx}{\linewidth}{XX}
    {
      boundary conditions at $s=0$:
      \begin{itemize}
        \item $y(0) = 0$
        \item $\od{y}{s}(0) = 0$
      \end{itemize}
    } & {
      boundary conditions at $s=L$:
      \begin{itemize}
        \item $v(L) = P$
        \item $M(L) = 0 \Rightarrow \od[2]{y}{s}(L) = 0$
      \end{itemize}
    }
  \end{tabularx}
\end{frame}

%-------------------------------------------------------------------------------
\begin{frame}
  \frametitle{Example: Cantilever beam (part 2)}
  
  \begin{figure}
    \centering
    \includegraphics[width=14cm, keepaspectratio=true]{sections/traditional_beams/images/EulerCanitleverExample1part2}
  \end{figure} 
  \vspace{-1.4em}
  
  \begin{displaymath}
  \text{moment balance:} \quad  -M(s) + P(L-s) = 0 \quad \Rightarrow \quad M(s) = P \, (L-s)
  \end{displaymath}
  
  \begin{displaymath}
    M(x) = E \, I \, \od[2]{y}{s} \quad \Rightarrow \quad \od[2]{y}{s} = \frac{M(s)}{E \, I}  = \frac{P \, (L-s)}{E \, I}
  \end{displaymath}
  
  \begin{displaymath}
    \Rightarrow y(s) = \frac{P}{E \, I} \, \biggl( - \frac{s^3}{6} + \frac{L \, s^2}{2} \biggr) + \cancel{c_1} \,s + \cancel{c_2} = \frac{P \, L^3}{2 \, E \, I} \Biggl( \biggl( \frac{s}{L} \biggr)^2 - \frac{1}{3} \biggl( \frac{s}{L} \biggr)^3 \Biggr)
  \end{displaymath}
\end{frame}

%-------------------------------------------------------------------------------
\begin{frame}
  \frametitle{Another example of a cantilever beam}
  
  \begin{figure}
    \centering
    \includegraphics[width=16cm, keepaspectratio=true]{sections/traditional_beams/images/EulerCanitleverExample2}
  \end{figure}
  
  \begin{tabularx}{\linewidth}{XX}
    {
      boundary conditions at $s=0$:
      \begin{itemize}
        \item $y(0) = 0$
        \item $\od{y}{s}(0) = 0$
      \end{itemize}
    } & {
      boundary conditions at $s=L$:
      \begin{itemize}
        \item $v(L) = P$
        \item $\od{y}{s}(L) = 0 \Rightarrow M(L) \neq 0$ \newline
          \null \quad (rotation is restricted)
      \end{itemize}
    }
  \end{tabularx}
  
  \begin{displaymath}
    \Rightarrow M(s) = M(L) + P \, (L-s)
  \end{displaymath}
\end{frame}


%-------------------------------------------------------------------------------
\begin{frame}
  \frametitle{Boundary conditions}
  
  \vspace{4em}
  \begin{center}
    \begin{tabular}{lr}
      displacement & force \\
      \hline
      prescribed $\Rightarrow$ & unknown \\
      unknown & $\Leftarrow$ prescribed \\
      \hline \\
      \\
      rotation & moment \\
      \hline
      prescribed $\Rightarrow$ & unknown \\
      unknown & $\Leftarrow$ prescribed \\
      \hline
    \end{tabular}
  \end{center}
\end{frame}


%===============================================================================
\subsection{Timoshenko beam theory}


%-------------------------------------------------------------------------------
\begin{frame}
  \frametitle{The Timoshenko beam (planar case)}
  
  \vspace{-1em}
  \begin{figure}
    \centering
    \includegraphics[width=16cm, keepaspectratio=true]{sections/traditional_beams/images/TimoshenkoBeam1}
  \end{figure}
  
  \begin{itemize}
    \item individual cross sections still remain planar
    \item centerline tangent orientation (red) $\neq$ cross section normal (green)
      \begin{itemize}
        \item $-\Theta$ measures orientation of cross section
        \item centerline tangent $\od{y}{s}$ influenced by \textbf{bending and shearing}
      \end{itemize}
    \item shear strain $\gamma(s) = \od{y}{s}(s) - \Theta(s) = \frac{v(s)}{k \, G \, A}$
    \item curvature $\kappa(s) = \od{\Theta}{s}(s) = \frac{M(s)}{E \, I}$ (see next slide)
  \end{itemize}
\end{frame}


%-------------------------------------------------------------------------------
\begin{frame}
  \frametitle{Connection between curvature and moment}

  \begin{multicols}{2}
    \noindent
    
    \begin{figure}
    \centering
      \includegraphics[width=13cm, keepaspectratio=true]{sections/traditional_beams/images/TimoshenkoBeam2}
    \end{figure}
    
    \begin{displaymath}
      L = R \, \Theta \quad \Rightarrow \quad \frac{1}{R} = \frac{\Theta}{L} = \od{\Theta}{s}
    \end{displaymath}
    
    \vspace{0.5em}
    fitting circles to the centerline locally:
    \begin{displaymath}
      \kappa(s) = \od{\Theta}{s}(s)
    \end{displaymath}
    
    \begin{displaymath}
      \epsilon_{xx} = \frac{\Delta L}{L} = \frac{(R-y) \, \Theta - R \, \Theta}{R \, \Theta} = \frac{-y}{R}
    \end{displaymath}
    
    \begin{displaymath}
      \sigma_{xx} = -E \, \frac{y}{R}
    \end{displaymath}
    
    \begin{displaymath}
      M = - \int_{\Omega} y \, \sigma_{xx} \, \dif A = E \, I \, \frac{1}{R} = E \, I \, \kappa
    \end{displaymath}
  \end{multicols}
\end{frame}


%-------------------------------------------------------------------------------
\begin{frame}
  \frametitle{Bending and axial compression together}

  \begin{multicols}{2}
    \noindent
    
    \begin{figure}
    \centering
      \includegraphics[width=13cm, keepaspectratio=true]{sections/traditional_beams/images/TimoshenkoBeam3}
    \end{figure}
      
    Because of axial forces the beam got shorter!
    \begin{displaymath}
      \begin{alignedat}{1}
        \epsilon_{xx} &= \frac{(r-y) \, \Theta - R \, \Theta}{R \, \Theta} = \frac{r-R}{R} + \frac{-y}{R} \\ 
        &= \epsilon - \frac{y}{R}
      \end{alignedat}
    \end{displaymath}
    
    \vspace{0.2em}
    stress due to axial stretch + bending:
    \begin{displaymath}
      \sigma_{xx} = E \, \bigl( \epsilon + \frac{-y}{R} \bigr)
    \end{displaymath}
    
    \vspace{-0.7em}
    \begin{displaymath}
      \begin{alignedat}{1}
        M &= - \int_{\Omega} y \, \sigma_{xx} \, \dif A \\
          &= - E \, \epsilon \cancel{\int_{\Omega} y \, \dif A} + \frac{E}{R} \int_{\Omega} y^2 \dif A \\ 
          &= E \, I \, \kappa
      \end{alignedat}
    \end{displaymath}
    
    $\rightarrow$ Bending and axial stretch are independent!
  \end{multicols}
\end{frame}


%-------------------------------------------------------------------------------
\begin{frame}
  \frametitle{Model linearity}
  
  \begin{displaymath}
    \begin{alignedat}{1}
      \od{y}{s} &= \frac{1}{k \, G \, A} \cdot v(s) + \Theta  \\ \\
      \od{\Theta}{s} &= \frac{1}{E \, I} \cdot M(s)
    \end{alignedat}
  \end{displaymath}

  \vspace{1em}
  We obtained a linear model because we assumed...
  \begin{itemize}
    \item small slopes of the centerline (linear kinematics)
      \begin{itemize}
        \item $\od{y}{s} = \tan{\Theta} \approx \Theta$
        \item approximation is not valid for larger deformations
      \end{itemize} 
    \item material linearity (linear constitutive law)
      \begin{itemize}
        \item $v(s)$, $M(s)$ get multiplied by constants \newline
          or by functions that depend on $s$ but not on $v(s)$, $M(s)$
        \item material may eg. get stiffer with increasing deformation energy
        \item btw. $k$ is a correction factor % TODO correction factor depends on ...
      \end{itemize}
  \end{itemize}
\end{frame}

%-------------------------------------------------------------------------------
\begin{frame}
  \frametitle{Example: Cantilever beam (part 1)}
  
  \begin{figure}
    \centering
    \includegraphics[width=16cm, keepaspectratio=true]{sections/traditional_beams/images/EulerCanitleverExample1part1}
  \end{figure}
  
  \begin{tabularx}{\linewidth}{XX}
    {
      Boundary conditions at $s=0$:
      \begin{itemize}
        \item $y(0) = 0$
        \item $\xcancel{\od{y}{s}(0) = 0}$ (because of shear!)
        \item $\Theta(0) = 0$ (cross section fixed)
      \end{itemize}
    } & {
      Boundary conditions at $s=L$:
      \begin{itemize}
        \item $v(L) = P$
        \item $M(L) = 0$
      \end{itemize}
    }
  \end{tabularx}
\end{frame}

%-------------------------------------------------------------------------------
\begin{frame}
  \frametitle{Example: Cantilever beam (part 2)}
  
  As before with the Euler-Bernoulli cantilever beam we have
  \begin{displaymath}
    v(s) = P \quad \text{ and } \quad M(s) = P \, (L-s)
  \end{displaymath}
  
  \vspace{1em}
  We integrate the Timoshenko beam equations ...
  \vspace{0.3em}
  \begin{displaymath}
    \od{\Theta}{s}(s) = \frac{M(s)}{E \, I} = \frac{P \, (L-s)}{E \, I}
    \quad \Rightarrow \quad
    \Theta(s) = \frac{P}{E \, I} \biggl( L \, s - \frac{s^2}{2} \biggr) + \cancel{c}
  \end{displaymath}
  
  \vspace{1em}
  
  \begin{displaymath}
    \od{y}{s}(s) = \frac{v(s)}{k \, G \, A} + \Theta(s) = \frac{P}{k \, G \, A} + \frac{P}{E \, I} \biggl( L \, s - \frac{s^2}{2} \biggr)
  \end{displaymath}
  
  \begin{displaymath}
    \Rightarrow y(s) = \frac{P \, s}{k \, G \, A} + \frac{P}{E \, I} \biggl( L \frac{s^2}{2} - \frac{s^3}{6} \biggr) + \cancel{c} = \frac{P \, s}{k \, G \, A} + \frac{P \, L^3}{2 \, E \, I} \Biggl( 
      \biggl( \frac{s}{L} \biggr)^2 - \frac{1}{3} \biggl( \frac{s}{L} \biggr)^3 \Biggr)
  \end{displaymath}
\end{frame}


%-------------------------------------------------------------------------------
\begin{frame}
  \frametitle{Comparison of Euler-Bernoulli and Timoshenko cantilever beams}
  
  \textbf{Euler-Bernoulli beam}
  \begin{displaymath}
    y^{(E)}(s) = \frac{P \, L^3}{2 \, E \, I} \Biggl( \biggl( \frac{s}{L} \biggr)^2 - \frac{1}{3} \biggl( \frac{s}{L} \biggr)^3 \Biggr)
    \Rightarrow y^{(E)}(L) = \frac{P \, L^3}{3 \, E \, I}
  \end{displaymath}
  
  \textbf{Timoshenko beam}
  \begin{displaymath}
    y^{(T)}(s) = \frac{P \, s}{k \, G \, A} + \frac{P \, L^3}{2 \, E \, I} \Biggl( 
      \biggl( \frac{s}{L} \biggr)^2 - \frac{1}{3} \biggl( \frac{s}{L} \biggr)^3 \Biggr)
      \Rightarrow y^{(T)}(L) = \frac{P \, L}{k \, G \, A} + \frac{P \, L^3}{3 \, E \, I}
  \end{displaymath}
  
  \textbf{relative error} (taking Euler-Bernoulli as the reference)
  \begin{displaymath}
    \begin{alignedat}{1}
      \epsilon_r &= \frac{y^{(T)}(L) - y^{(E)}(L)}{y^{(E)}(L)} = \frac{\frac{P \, L}{k \, G \, A}}{\frac{P \, L^3}{3 \, E \, I}} = \frac{3 \, E \, I}{k \, G \, A \, L^2} \\
      &= \frac{3 \, E \, b \, h^3}{k \, G \, b \, h \, L^2 \, 12} = \frac{1}{4} \frac{E}{k \, G} \biggl( \frac{h}{L} \biggr)^2
    \end{alignedat}
  \end{displaymath}
  
  \begin{center}
    $\rightarrow$ Euler-Bernoulli beam model performs equally good as the Timoshenko model \newline when the beam is very slender ($h/L$ small)
  \end{center}
\end{frame}


%-------------------------------------------------------------------------------
\begin{frame}
  \frametitle{Another example of a cantilever beam (part 1)}
  
  \vspace{-0.7em}
  \begin{figure}
    \centering
    \includegraphics[width=16cm, keepaspectratio=true]{sections/traditional_beams/images/TimoshenkoCanitleverExample2part1}
  \end{figure}
  \vspace{-0.5em}
  cross section at $s=L$ can not rotate or displace horizontally
  \vspace{1em}
  
  \begin{tabularx}{\linewidth}{XX}
    {
      boundary conditions at $s=0$:
      \begin{itemize}
        \item $y(0) = 0$
        \item $\Theta(0) = 0$
      \end{itemize}
    } & {
      boundary conditions at $s=L$:
      \begin{itemize}
        \item $v(L) = P$
        \item $\Theta(L) = 0$
      \end{itemize}
    }
  \end{tabularx}
\end{frame}


%-------------------------------------------------------------------------------
\begin{frame}
  \frametitle{Another example of a cantilever beam (part 2)}
  
  \vspace{-0.7em}
  \begin{figure}
    \centering
    \includegraphics[width=20cm, keepaspectratio=true]{sections/traditional_beams/images/TimoshenkoCanitleverExample2part2}
  \end{figure}
  
  moment balance (in the deformed configuration):
  \begin{displaymath}
    -M(s) + M(L) + P \, (L-s) - N(y(L)-y(s))
  \end{displaymath}
  $\rightarrow$ the last term results in nonlinear behavior of the model \newline
  $\rightarrow$ $M(L)$ and $N(L)$ are two extra unknowns \newline
  % TODO check this example
\end{frame}


%-------------------------------------------------------------------------------
% include section cosserat_rods
%-------------------------------------------------------------------------------
% Special Cosserat rods
%-------------------------------------------------------------------------------
\section{Theory of Special Cosserat rods}

%-------------------------------------------------------------------------------
\begin{frame}
  \frametitle{Introduction to \st{special} Cosserat rods (part 1)}
  \vspace{-1em}
  \begin{figure}
    \centering
    \includegraphics[width=20cm, keepaspectratio=true]{sections/cosserat_rods/images/Kinematics}
  \end{figure}
  
  Remember: $s$ is the unique identifier for a particular cross section of the beam
  \vspace{0.6em}
  
  Kinematic quantities:
  \begin{itemize}
    \item position of the deformed cross section $\underline{r}(s)$
    \item shearing of the deformed cross section $\underline{d}_1(s)$, $\underline{d}_2(s)$
    \item orientation of the deformed cross section $\underline{d}_3(s)$ \newline
      \null \quad (perpendicular to $\underline{d}_1(s)$ and $\underline{d}_2(s)$ $\rightarrow$ not independent)
    %\item $\doubleunderline{R}(s)$
  \end{itemize}
  
\end{frame}


%-------------------------------------------------------------------------------
\begin{frame}
  \frametitle{Introduction to \st{special} Cosserat rods (part 2)}

  Constrained deformation map:
  \begin{displaymath}
    \underline{f}(\underline{X}) = \underline{f}(X_1,X_2,X_3=s) = \underline{r}(s) + X_{\alpha} \, \underline{d}_{\alpha} \quad \text{with } \alpha \in \{1,2\}
  \end{displaymath}
  
  Consequences:
  \begin{itemize}
    \item cross sections remain flat
    \item straight lines within the cross section remain straight lines
    \item boundary of the cross section: circles are mapped to ellipses \newline
      \null \quad (not to arbitrary  curves in 2D)
    \item $\underline{d}_1(s)$ and $\underline{d}_2(s)$ are not perpendicular in the general case (shearing)
  \end{itemize}

\end{frame}



%-------------------------------------------------------------------------------
\begin{frame}
  \frametitle{Kinematics of the Special Cosserat rod}

  What makes the Special Cosserat rod special?:
  \begin{itemize}
    \item $\underline{d}_1(s)$ and $\underline{d}_2(s)$ \textit{are} perpendicular and unit-normed
    \item hence $\left( \underline{d}_1, \underline{d}_2, \underline{d}_3 \right)$ form an orthonormal triad
  \end{itemize}
  \vspace{0.6em}
  
  Consequences:
  \begin{displaymath}
    \exists \: \doubleunderline{R}(s) \in SO3 \: \forall s : \biggl( R: \left( \underline{e}_1, \underline{e}_2, \underline{e}_3 \right) \mapsto \left( \underline{d}_1, \underline{d}_2, \underline{d}_3 \right) \biggr)
  \end{displaymath}
  % TODO SO3 (twice)
  \begin{itemize}
    \item $\left( \underline{e}_1, \underline{e}_2, \underline{e}_3 \right)$ is called the global basis
    \item $\left( \underline{d}_1, \underline{d}_2, \underline{d}_3 \right)(s)$ is called the local basis or director basis at $s$
    \item $\underline{d}_i(s) = \doubleunderline{R}(s) \cdot \underline{e}_i$ \quad 
      ($\doubleunderline{R}$ : rotation matrix ; $SO3$ : special orthogonal matrix group)
  \end{itemize}
  \vspace{0.6em}
  
  Constrained deformation map:
  \begin{displaymath}
    \underline{f}(\underline{X}) = \underline{f}(X_1,X_2,X_3=s) = \underline{r}(s) + \doubleunderline{R}(s) \cdot (X_{\alpha} \, \underline{e}_{\alpha}) \quad \text{with } \alpha \in \{1,2\}
  \end{displaymath}
  
  \vspace{0.5em}
  Remark: The rigidity of the cross section is stiffening the beam!
\end{frame}


%-------------------------------------------------------------------------------
\begin{frame}
  \frametitle{Kinematics of the Special Cosserat rod (warp-mode on ;-)}
  
  MESSED UP:
  Remember: more accuracy $\rightarrow$ more unknowns
  \vspace{0.6em}
  % TODO check again from here , extra unknown
  We introduce warping of the cross section, 
  while at the same time keeping the director basis $\underline{d}_i$. Modified deformation map:
  \begin{displaymath}
    \underline{f}(\underline{X}) = \underline{f}(X_1,X_2,X_3=s) = \underline{r}(s) + \doubleunderline{R}(s) \cdot (X_{\alpha} \, \underline{e}_{\alpha} + \underline{u}) \quad \text{with } \alpha \in \{1,2\}
  \end{displaymath}
  
  in-plane: shrinking
  out-of-plane: warping, not planar
  
  if $\underline{u}(X_1,X_2)$ then we would again deal with a 3D elasticity problem \newline
  % small x?
  $\rightarrow$ $\underline{u}$ depends on local measures (1D theory)
  $\underline{u}$ is not an independent quantity / extra unknown
  
  $\underline{u}(X_1,X_2,\text{local strains})$
  local strains: local gradients of $\underline{r}$ and $\doubleunderline{R}$
  
  $\underline{d}_1$, $\underline{d}_1$ are perpendicular and represent the average orientation of the warped cross section
  
\end{frame}



%-------------------------------------------------------------------------------
\begin{frame}
  \frametitle{...}
  
  % TODO order of slides
  3D elasticity , plug in the constrained deformation map , integrate over cross section , obtain a system of ODEs in the unknowns $\underline{r}(s)$ and $\doubleunderline{R}(s)$
  
\end{frame}

%-------------------------------------------------------------------------------
\begin{frame}
  \frametitle{Rotations in 3D revisited}
  
  A rotation about \textit{one} axis is determined by
  \begin{itemize}
    \item axis of rotation given by $\underline{a}$ with $\norm{\underline{a}} = 1$
    \item angle of rotation $\Theta$
  \end{itemize}
  \vspace{0.6em}
  
  Composition of rotations:
  \begin{displaymath}
    \left( \underline{a}_1, \Theta_1 \right) + \left( \underline{a}_2, \Theta_2 \right) + \left( \underline{a}_3, \Theta_3 \right) + \dots = \left( \underline{a}_{eff}, \Theta_{eff} \right)
  \end{displaymath}
  \begin{displaymath}
    \doubleunderline{R}_{eff} = \dots \cdot \doubleunderline{R}_3 \cdot \doubleunderline{R}_2 \cdot \doubleunderline{R}_1
  \end{displaymath}
  Remark: In the general case ($\underline{a}_i \neq \underline{a}_j \text{ for } i \neq j$) rotations do not commute!
  \vspace{1em}
  
  Example (rotation about $\underline{e}_3$):
  \begin{displaymath}
    \Biggl( \underline{a} =
    \begin{bmatrix}
      0 \\ 0 \\ 1
    \end{bmatrix}, \Theta \Biggr) \quad \rightarrow \quad
    \doubleunderline{R} =
    \begin{bmatrix}
      +\cos(\Theta) & -\sin(\Theta) & 0 \\
      +\sin(\Theta) & +\cos(\Theta) & 0 \\
      0 & 0 & 1
    \end{bmatrix}
  \end{displaymath}

\end{frame}

%-------------------------------------------------------------------------------
\begin{frame}
  \frametitle{Axis-angle representation of rotations in 3D}
  
  From the axis-angle representation
  \begin{displaymath}
    \Biggl( \underline{a} =
    \begin{bmatrix}
      a_1 \\ a_2 \\ a_3
    \end{bmatrix}, \Theta \Biggr)
  \end{displaymath}
  we obtain the corresponding rotation matrix
  \begin{displaymath}
    \doubleunderline{R} = \exp(\Theta \cdot \doubleunderline{a})  
  \end{displaymath}
  with
  \begin{displaymath}
    \Theta \cdot \doubleunderline{a} = \Theta \cdot 
    \begin{bmatrix}
      0 & -a_3 & a_2 \\
      a_3 & 0 & -a_1 \\
      -a_2 & a_1 & 0
    \end{bmatrix}
  \end{displaymath}
  a skew-symmetric matrix obtained from the definition $\doubleunderline{a} = \doubleunderline{a} \cdot \doubleunderline{I} = \underline{a} \times \doubleunderline{I}$
  
  \vspace{2em}
  Observe the isomorphism between cross products and skew-symmetric matrices:
  \begin{displaymath}
    \underline{a} = \axial(\doubleunderline{a}) = \axial(\underline{a} \times \doubleunderline{I}) = \axial([\underline{a}]_{\times})
  \end{displaymath}
\end{frame}


%-------------------------------------------------------------------------------
\begin{frame}
  \frametitle{Axis-angle representation of rotations in 3D: Rodrigues' formula}
  
  The Rodrigues' rotation formula allows us to compute the rotation matrix $\doubleunderline{R}$ that corresponds to a given axis-angle representation $\left( \underline{a}, \Theta \right)$ without actually computing the matrix exponential
  
  \begin{displaymath}
    \doubleunderline{R} = \cos(\Theta) \, \doubleunderline{I} + \sin(\Theta) \, \doubleunderline{a} + (1-\cos(\Theta)) \, \underline{a} \otimes \underline{a}
  \end{displaymath}
  
  Example:
  \begin{displaymath}
    \underline{v}_{\text{rotated}} = \doubleunderline{R} \, \underline{v} = \cos(\Theta) \, \underline{v} + \sin(\Theta) \underline{a} \times \underline{v} + (1-\cos(\Theta)) \, (\underline{a} \cdot \underline{v}) \,\underline{a}
  \end{displaymath}
  
%  \begin{displaymath}
%    \doubleunderline{R} = \exp([\underline{a}]_{\times}) = \doubleunderline{I} + \sin(\norm{\underline{a}}) \biggl( \frac{\underline{a}}{\norm{\underline{a}}} \biggr)
%  \end{displaymath}

\end{frame}


%-------------------------------------------------------------------------------
\begin{frame}
  \frametitle{3D rotations expressed by unit quaternions}
  
  Quaternions are a number system that extends the complex numbers. A quaternion consists of one real part and three independent imaginary parts. A unit quaternion is a quaternion of norm one and therefore has three independent components. There exists an isomorphism between unit quaternions and rotation matrices
  % TODO check if formulas are valid? more info on this stuff useful?
  \begin{displaymath}
    \text{given }
    \underline{q} = \begin{bmatrix}
      q_0 \\ q_1 \\ q_2 \\ q_3
    \end{bmatrix} \quad \rightarrow \quad
    \doubleunderline{R}(\underline{q}) = 2 \cdot \begin{bmatrix}
      \frac{1}{2} - (q_2^2 + q_3^2) & q_1 \, q_2 - q_0 \, q_3 & q_1 \, q_3 + q_0 \, q_2 \\
      q_1 \, q_2 + q_0 \, q_3 & \frac{1}{2} - (q_1^2 + q_3^2) & q_2 \, q_3 - q_0 \, q_1 \\
      q_1 \, q_3 - q_0 \, q_2 & q_2 \, q_3 + q_0 \, q_1 & \frac{1}{2} - (q_1^2 + q_2^2)
    \end{bmatrix}
  \end{displaymath}
  
  \begin{displaymath}
    \text{real part: } q_0 = \cos \biggl( \frac{\Theta}{2} \biggr) \: \text{ and imaginary part: } \begin{bmatrix}
      q_1 \\ q_2 \\ q_3
    \end{bmatrix} = \sin \biggl( \frac{\Theta}{2} \biggr) \, \underline{a}
  \end{displaymath}
  
  advantage: quadratic polynomials are much faster for computation than $\sin(.)$ and $\cos(.)$

\end{frame}







%-------------------------------------------------------------------------------
% include section ...
%\include{sections/blueprint/blueprint}



%-------------------------------------------------------------------------------
\end{document}
