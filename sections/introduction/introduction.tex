%===============================================================================
\section*{Introduction}


%-------------------------------------------------------------------------------
\begin{frame}
  \frametitle{Extension, torsion and bending of rods/beams}
  
  \begin{tabularx}{\linewidth}{XX}
    {
    \begin{alignat*}{1}
      F &= E \, A \, \epsilon \\
        %&= E \, A \, \frac{\Delta l}{l}
    \end{alignat*}
    } & {
    \null
    } \\
    %\hline
    
    {
    \begin{alignat*}{1}
      M_t = G \, J \, \frac{\Theta}{L} \\
    \end{alignat*}
    } & {
    \null
    } \\
    %\hline
        
    {
    \begin{alignat*}{1}
      M = E \, I \, \kappa \\
    \end{alignat*}
    } & {
    \null
    }
  \end{tabularx}  
  
  % TODO revisit this/these motivational slide/slides
  % put in three pictures
\end{frame}


%-------------------------------------------------------------------------------
\begin{frame}
  \frametitle{3D elasticity (recap of some basics)}
  
  \textbf{deformation map} $\underline{x} = \underline{f}(\underline{X})$
  \begin{itemize}
    \item takes a (Langrangian/material) vector/point $\underline{X}$ in the undeformed configuration
    \item returns a (Eulerian/spatial) vector/point $\underline{x}$ in the deformed configuration
  \end{itemize}
  \vspace{1em}
  
  \textbf{deformation gradient} $\doubleunderline{F}(\underline{X}) = \nabla_{\underline{X}} \,\underline{f}(\underline{X})$
  \begin{itemize}
    \item takes a (Langrangian/material) vector/point $\underline{X}$ in the undeformed configuration \newline
      $\rightarrow$ $\underline{X}$ is material point where the gradient is evaluated
    \item returns a second order (two-point) tensor $\doubleunderline{F}$ that depends on $\underline{X}$ (in the general case)
    \item $\doubleunderline{F}$ contains (first order accurate) information about how line elements in the material configuration are transformed into deformed (stretched and rotated) line elements in the spatial configuration by $\underline{f}(.)$ \newline
      $\rightarrow$ $\dif \underline{x} = \doubleunderline{F}(\underline{X}) \cdot \dif \underline{X}= \nabla_{\underline{X}} \,\underline{f}(\underline{X}) \cdot \dif \underline{X}$
  \end{itemize}
  \vspace{1em}
  
  \textbf{equations of equilibrium}
  \begin{itemize}
    \item material description: $\nabla_{\underline{X}} \cdot \doubleunderline{P} + \underline{B} = \rho_0 \cdot \underline{A}$
    \item spatial description: $\nabla_{\underline{x}} \cdot \doubleunderline{\sigma} + \underline{b} = \rho \cdot \underline{a}$
  \end{itemize}
\end{frame}


%-------------------------------------------------------------------------------
\begin{frame}
  \frametitle{1D elasticity}
  
  we talk about \textbf{slender bodies} ($\frac{\mathrm{Length}}{\mathrm{Diameter}} > 10$)
  \vspace{1em}
  
  we only have \textbf{one material coordinate} $s$ along the axis of the beam
  \begin{itemize}
    \item $s$ uniquely identifies a particular cross section along the beam axis
    \item model equations will be a system of ODEs in $s$ (instead of PDE)
    \item equations are obtained by averaging a PDE in 3 dimensions over the cross section
  \end{itemize}
  \vspace{1em}
  
  kinematic quantities in the model
  \begin{itemize}
    \item \textbf{position of centerline} $\underline{r}(s)$
    \item \textbf{orientation/rotation of cross section} $\doubleunderline{R}(s)$
    \item more accuracy $\rightarrow$ more unknowns
  \end{itemize}
\end{frame}
